In the previous chapters, we developed agents with very simple genes.
However, genes can be arbitrarily complex.
The default implementation of |Gene| is usually sufficient,
as in the following example.

\begin{code}
data ComplexGene = A | B Colour | C Word8 | D Bool Char | E [ComplexGene]
  deriving (Show, Eq, Generic)

instance Genetic ComplexGene
\end{code}

You are not restricted to using the default implementation of |Gene|.
In the following example, we write a custom implementation.
You can mix custom and default implementations;
the implementation of |Gene| for |CustomGene|
uses the default implementation of |Gene| for |Colour|.

\begin{code}
data CustomGene = F Colour | G Bool
  deriving (Show, Eq, Generic)

instance Genetic CustomGene where
  put (F c) = putRawWord8 7 >> put c
  put (G b) = putRawWord8 8 >> put b
  get = do
    x <- getRawWord8
    case x of
      (Right 7) -> do
        c <- get :: Reader (Either [String] Colour)
        return . fmap F $ c
      (Right 8) -> do
        b <- get :: Reader (Either [String] Bool)
        return . fmap G $ b
      _      -> return $ Left ["Invalid gene sequence"]
\end{code}

One reason you might want to write a custom implementation of |Gene|
is for efficiency.
In this example, we store three boolean values in a Word8 value
to reduce the amount of storage required.
      
\begin{code}
data CustomGene2 = H Bool Bool Bool
  deriving (Show, Eq, Generic)

instance Genetic CustomGene2 where
  put (H x y z) = putRawWord8 (x' + y' + z')
    where x' = (4 *) . fromIntegral . fromEnum $ x :: Word8
          y' = (2 *) . fromIntegral . fromEnum $ y :: Word8
          z' = fromIntegral . fromEnum $ z :: Word8
  get = do
    w <- getRawWord8 :: Reader (Either [String] Word8)
    let x = fmap (flip testBit 2) w :: Either [String] Bool
    let y = fmap (flip testBit 1) w :: Either [String] Bool
    let z = fmap (flip testBit 0) w :: Either [String] Bool
    return $ H <$> x <*> y <*> z
\end{code}

You will find these examples in the code listing below.
To run the example, type |chapter11|.
\includeSource{src/Tutorial/Chapter11/Example.hs}
