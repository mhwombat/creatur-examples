
The module \path{ALife.Creatur.Universe.Task} provides some tasks
that you can use with the daemon.
These tasks handle reading and writing agents,
which reduces the amount of code you need to write.
(It's also easy to write your own tasks, using those in 
\path{ALife.Creatur.Universe.Task} as a guide.)
The simplest task has the type signature:

\begin{code}
runNoninteractingAgents ∷ (Clock c, Logger l, Database d, Agent a, 
    Serialize a, Record a, a ~ DBRecord d) ⇒
  AgentProgram c l d n x a → StateT (Universe c l d n x a) IO ()
\end{code}

That signature looks complex, but essentially it means that if you
supply an |AgentProgram|, it will run it for you.
Your |AgentProgram| will run in a universe that implements
\path{ALife.Creatur.AgentNamer},
\path{ALife.Creatur.Clock}, and
\path{ALife.Creatur.Logger}.
So what type signature must your |AgentProgram| have?
Here's the definition.

\begin{code}
type AgentProgram c l d n x a = a → StateT (Universe c l d n x a) IO a
\end{code}

This is the signature for an agent that doesn't interact with other
agents.
The input parameter is the agent whose turn it is to use the CPU.
The program must return the agent (which may have been modified).
The universe will then automatically be updated with these changes.
We will use the |run| function in \path{Tutorial.Example1.Rock}
(discussed in Section \ref{sec:species1}) as our |AgentProgram|.
Its type signature is shown below.

\begin{code}
run :: Rock -> StateT (SimpleUniverse a) IO Rock
\end{code}

|SimpleUniverse a| provides the logging, clock, and agent naming
functionality we need, and satisfies |Universe c l d n x a|.
So the type signature of |run| is consistent with |AgentProgram|.
The program below configures and launches the daemon.

\includeSource{src/Tutorial/Example1/Daemon.hs}

