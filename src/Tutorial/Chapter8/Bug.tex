We create a type to represent the |Bug| species.
Each bug has a unique ID, a colour, sex, an energy level,
and some genetic information.
Because our bugs will reproduce \emph{sexually}, they have \emph{two} sets of genes, 
one inherited from each parent.
Thus, the genome will be stored as
a |PairedSequence| instead of a |Sequence|.

\begin{code}
data Bug = Bug
  { 
    bugName :: String,
    bugColour :: BugColour,
    bugSex :: Sex,
    bugEnergy :: Int,
    bugGenome :: DiploidSequence
  } deriving (Show, Generic)

instance Serialize Bug
\end{code} 

We make |Bug| implement the |Agent| 
and |Record| classes.
Our bugs will stay alive until their energy reaches zero.

\begin{code}
instance Agent Bug where
  agentId = bugName
  isAlive bug = bugEnergy bug > 0

instance Record Bug where key = agentId
\end{code} 

We create the genes for colour and sex.

\begin{code}
data BugColour = Green | Purple
  deriving (Show, Eq, Enum, Bounded, Generic)
instance Serialize BugColour
instance Genetic BugColour

data Sex = Male | Female
  deriving (Show, Eq, Enum, Bounded, Generic)
instance Serialize Sex
instance Genetic Sex
\end{code} 

Recall that our bugs have \emph{two} sets of genes.
These genes may not be identical,
so we need a way to determine the resulting colour of the bug
from its genes.
The |Diploid| class contains a method called |generic|, which, 
given two possible forms of a gene, takes into account any dominance relationship,
and returns a gene representing the result.

We could write an implementation of |express|,
but the |Diploid| class provides a generic implementation.
The generic implementation of |express| chooses the "smaller" of the two values.
For numeric values, this simply means taking the minimum of the two values,
so a gene with a value of |3.5| is dominant over one with a value of |7.6|.
For types with multiple constructors, the constructors that appear
earlier in the definition are dominant over those that appear later,
so a |Male| gene will be dominant over a |Female| gene.
In other words, a bug with two |Female| genes is female, but a bug with at least one 
|Male| gene is male.
This is loosely based on the XY-chromosome system used by
humans and some other animals.
(To avoid giving the males too much power, we'll let the females initiate mating!)

\begin{code}
instance Diploid BugColour
instance Diploid Sex
\end{code} 

To support reproduction, we need a way to build a bug from its genome.
Like the plants we created earlier, each bug needs a copy of its genome in order to produce offspring.
For plants, we used the method |copy| to get the unread genetic information.
However, bugs have two sets of genes, so we use |copy2| instead.
Next, we determine the sex and colour of the bug from its genome.
We could use the method |getAndExpress| in the module |BRGCBool|, 
which returns a |Maybe| value containing a tuple with the first gene in a sequence,
and the rest of the paired sequences.
It may happen that neither sequence of |Bool|s begins with
a valid code for a colour or for the sex, in which case the call to |getAndExpress|
would return |Nothing|.
For convenience, we'll use the |getAndExpressWithDefault| method instead,
supplying a default sex and colour.
All bugs start life with 10 units of energy.
\begin{code}
buildBug :: String -> DiploidReader (Either [String] Bug)
buildBug name = do
  g <- copy2
  sex <- getAndExpressWithDefault Female
  colour <- getAndExpressWithDefault Green
  return . Right $ Bug name colour sex 10 g
\end{code}

Next, we need a way to mate two bugs and produce some offspring.
We can do this by implementing the |Reproductive| class in 
\path{ALife.Creatur.Genetics.Reproduction.Sexual}.
This class requires us to implement the following:
\begin{enumerate}
\item A type called |Base|, which specifies the type used to encode
genes for this species. Recall that we've used |Bool|s for this purpose.
\item A method called |produceGamete| which shuffles (and maybe mutates)
the two sequences of genes from one parent, 
and then produces a \emph{single} sequence that will become part of the
child's genome.
(This is analogous to creating either a single sperm or ova.)
\item A method called |build| which creates the offspring.
We can use the |buildBug| method that we've already created,
and pass it to |runReader|.
\end {enumerate}

\begin{code}
instance Reproductive Bug where
  type Base Bug = Sequence
  produceGamete a = 
    repeatWithProbability 0.1 randomCrossover (bugGenome a) >>=
    withProbability 0.01 randomCutAndSplice >>=
    withProbability 0.001 mutatePairedLists >>=
    randomOneOfPair
  build name = runDiploidReader (buildBug name)
\end{code} 

Although the implementation of |produceGamete| is similar to that
of |recombine| for the |Plant| class, 
these two functions have different uses.
Asexual reproduction uses |recombine| to mix the genetic
information from two parents;
the resulting sequence becomes the entire genome of the child.
Sexual reproduction uses |produceGamete| to mix the genetic information
from \emph{one} parent;
the resulting sequence becomes \emph{half} of the genome of the child.
The other half of the child's genome comes from the other parent,
also generated using |produceGamete|.

The function |run| is invoked when it is the agent's turn to use the CPU.
It takes a list of all agents in the population, with the current agent at the head of the list.
It returns a list of agents that have been created or modified during the turn.
Let's let the females initiate mating.
If this bug is female, and the second bug is male, then mating occurs.
If mating occurs, we deduct one unit of energy.
If mating does not occur, then no agents have been modified so we return an empty list.

\begin{code}
run :: [Bug] -> StateT (SimpleUniverse Bug) IO [Bug]
run (me:other:_) = do
  writeToLog $ agentId me ++ "'s turn" 
  if bugSex me == Female && bugSex other == Male
    then do
      name <- genName
      (Right baby) <- liftIO $ evalRandIO (makeOffspring me other name)
      writeToLog $ 
        bugName me ++ " and " ++ bugName other ++
          " gave birth to " ++ name ++ ", a " ++ 
          show (bugColour baby) ++ " " ++ show (bugSex baby) ++ " bug"
      return [deductMatingEnergy me, deductMatingEnergy other, baby]
    else return []
run _ = return [] -- need two agents to mate

deductMatingEnergy :: Bug -> Bug
deductMatingEnergy bug = bug {bugEnergy=bugEnergy bug - 1}
\end{code}

The complete code listing is below.

\includeSource{src/Tutorial/Chapter8/Bug.hs}

